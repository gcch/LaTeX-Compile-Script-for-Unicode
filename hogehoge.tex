\documentclass[a4paper,12pt]{ujreport}

% 文字関連
\usepackage{newtxtext}% newtx (Text): Times
\usepackage[libertine]{newtxmath}% newtx (Math): Libertine
\usepackage[scaled=.95]{helvet}% Helvetica (95%)
\renewcommand{\ttdefault}{pcr}% Courier

% 数式関連
\usepackage{amsmath,amssymb}
\usepackage{mathtools}

% 図関連
\usepackage[dvipdfmx]{graphicx}
\graphicspath{{./figures/}}% 図のデフォルトフォルダパス

%%%%%%%%%%%%%%%%%%%%%%%%%%%%%%%%%%%%%%%%%%%%%%%%%%

\begin{document}

%%%%%%%%%%%%%%%%%%%%%%%%%%%%%%%%%%%%%%%%%%%%%%%%%%

\chapter*{Windows}

\section*{Serif + 明朝}
\begin{center}
Windows でコンピュータの世界が広がります.1234567890
\end{center}

\section*{Sans-serif + ゴシック}
\begin{center}
\sffamily\gtfamily
Windows でコンピュータの世界が広がります.1234567890
\end{center}

%%%%%%%%%%%%%%%%%%%%%%%%%%%%%%%%%%%%%%%%%%%%%%%%%%

\chapter*{Mac OS X}

\section*{Serif + 明朝}
\begin{center}
あのイーハトーヴォの \\
すきとおった風, \\
夏でも底に冷たさをもつ青いそら, \\
うつくしい森で飾られたモーリオ市, \\
郊外のぎらぎらひかる草の波. \\
祇辻飴葛蛸鯖鰯噌庖箸 \\
ABCDEFGHIJKLM \\
abcdefghijklm \\
1234567890 \\
\end{center}


\section*{Sans-serif + ゴシック}
\begin{center}
\sffamily\gtfamily
あのイーハトーヴォの \\
すきとおった風, \\
夏でも底に冷たさをもつ青いそら, \\
うつくしい森で飾られたモーリオ市, \\
郊外のぎらぎらひかる草の波. \\
祇辻飴葛蛸鯖鰯噌庖箸 \\
ABCDEFGHIJKLM \\
abcdefghijklm \\
1234567890 \\
\end{center}

%%%%%%%%%%%%%%%%%%%%%%%%%%%%%%%%%%%%%%%%%%%%%%%%%%

\end{document}
